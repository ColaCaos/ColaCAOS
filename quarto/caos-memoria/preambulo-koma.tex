% ====== COLORES Y TIPOS DE TÍTULO (KOMA nativo) ======
\usepackage{xcolor}
\definecolor{Main}{HTML}{0F766E}     % verde azulado
\definecolor{Accent}{HTML}{2563EB}   % azul acento

% Tipos para cada nivel KOMA
\setkomafont{part}{\normalfont\bfseries\Huge\color{Main}}
\setkomafont{chapter}{\normalfont\bfseries\Huge\color{Main}}
\setkomafont{section}{\normalfont\bfseries\Large\color{Main}}
\setkomafont{subsection}{\normalfont\bfseries\large\color{Accent}}
\setkomafont{subsubsection}{\normalfont\bfseries\normalsize\color{Accent}}

% Espaciados de títulos (más seguros con nuestra cabecera)
\RedeclareSectionCommand[
  beforeskip=-2.0ex plus -1ex minus -.2ex,
  afterskip=2.0ex plus .5ex
]{chapter}
\RedeclareSectionCommand[
  beforeskip=-1.5ex plus -.7ex minus -.2ex,
  afterskip=1.0ex plus .3ex
]{section}
\RedeclareSectionCommand[
  beforeskip=-1.0ex plus -.5ex minus -.2ex,
  afterskip=.8ex plus .2ex
]{subsection}

% ====== CABECERA DE CAPÍTULO (franja superior + regla bajo título) ======
\usepackage{etoolbox} % <— nuevo, para ifstrempty

% ====== CABECERA DE CAPÍTULO (franja superior + regla bajo título) ======
\usepackage{tikz}
\newcommand*\ChapterDecor{%
  \begin{tikzpicture}[remember picture,overlay]
    \fill[Main] (current page.north west) rectangle ([yshift=-1.2cm]current page.north east);
  \end{tikzpicture}%
}

% Mantengo tu formato de número
%\renewcommand*\chapterformat{{\Large\bfseries\color{Main}\thechapter}\enskip}
\renewcommand*\chapterformat{%
  {\fontsize{38}{40}\selectfont\bfseries\color{Main}\thechapter}\enskip
}
% Imprime título con o sin número (no número en capítulos * como ToC)
\renewcommand*\chapterlinesformat[3]{%
  \ChapterDecor%
  \vspace*{1.4cm}%
  {\raggedright
   {\Huge\bfseries\color{Main}%
    \ifstrempty{#2}{}{#2\enskip}% #2 vacío => capítulo sin numerar (no imprimir número)
    #3\par}
   \vspace{.25cm}
   {\color{Main}\rule{\textwidth}{1.2pt}}
  }%
  \par\vspace{1.2ex}%
}


% ====== ENCABEZADOS Y PIES (KOMA: scrlayer-scrpage) ======
\usepackage[automark]{scrlayer-scrpage}
\clearpairofpagestyles
\ohead{\headmark}
\cfoot{\pagemark}
\setkomafont{pagehead}{\normalfont\itshape\color{gray!60}}
\setkomafont{pagenumber}{\normalfont\bfseries\color{gray!70}}

% ====== FIGURAS: ajuste global y enlaces clicables ======
\usepackage{graphicx}
\setkeys{Gin}{width=.8\linewidth,height=7cm,keepaspectratio}
\usepackage{hyperref}
\hypersetup{
  colorlinks=true,
  linkcolor=Accent,
  citecolor=Accent,
  urlcolor=Accent
}

% ====== PAQUETES DE MATEMÁTICAS ======
\usepackage{amsthm,amsmath,amssymb}
\usepackage{scrhack}
%\everydisplay{\tag{\thechapter.\arabic{equation}}\stepcounter{equation}}

% ====== COMPATIBILIDAD con comandos antiguos (\rm, \bf, \it, \tt, \sc) ======
\makeatletter
\DeclareOldFontCommand{\rm}{\normalfont\rmfamily}{\mathrm}
\DeclareOldFontCommand{\bf}{\normalfont\bfseries}{\mathbf}
\DeclareOldFontCommand{\it}{\normalfont\itshape}{\mathit}
\DeclareOldFontCommand{\tt}{\normalfont\ttfamily}{\mathtt}
\DeclareOldFontCommand{\sc}{\normalfont\scshape}{\@nomath\sc}
\makeatother

% ====== NUMERACIÓN POR CAPÍTULO ======
\usepackage{chngcntr}
\numberwithin{equation}{chapter}
\counterwithin{figure}{chapter}
\counterwithin{table}{chapter}

% ====== REFERENCIAS ======
\usepackage[nameinlink,capitalise]{cleveref}

% ====== UNICODE ÚTIL EN TEXTO ======
\usepackage{newunicodechar}
\newunicodechar{⇒}{\ensuremath{\Rightarrow}}
\newunicodechar{≥}{\ensuremath{\ge}}
\newunicodechar{𝛿}{\ensuremath{\delta}}
\newunicodechar{𝜃}{\ensuremath{\theta}}

% ====== ECUACIONES LARGAS ======
\allowdisplaybreaks

\usepackage{etoolbox}
\makeatletter
\g@addto@macro\@floatboxreset\centering
\makeatother

\renewcommand{\contentsname}{Tabla de contenidos}